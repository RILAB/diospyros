\documentclass[]{article}

\begin{document}

\title{Whole-genome shotgun sequence of four Diospyros taxa}
\author{Dianne Velasco, MItchell C. Provance, Jeffrey Ross-Ibarra}
\date{Today}
\maketitle

The genus \emph{Diospyros} (Ebenaceae) is comprised of 500-600 species of trees and shrubs with a mainly pantropical-subtropical distribution. It is rich in useful species, being particularly well known for ebony, a dense, black, precious wood, produced by some species, and for delicious berries, sometimes called persimmons, which are produced by many species. Economically, the most important species include Japanese persimmon (D. kaki Thunb.) and Coromandel ebony (D. melanoxylon Roxb.). Based on FAO reports (2008), Sablok et al. (2011) reported 2,340,000 tons of Japanese persimmon fruit produced in 2007, mostly in China. In India, production of tendu, the leaves from Coromandel ebony that are used for wrapping bidi cigarettes, averages 300,000 tons annually, and employs 7.5 million harvesters for 3 months each year (Singh and Paul 2012). Because an unrelated, species-rich genus, Dalbergia (Fabaceae), contributes significantly to world ebony production, is largely sympatric with Diospyros, and because some segments of the precious wood industry operate clandestinely in order to circumvent CITES regulations, the importance of individual Diospyros taxa to the precious wood industry is difficult to assess. Nevertheless, it seems a large percentage of the known species of Diospyros find use in medicine, minor or locally important fruit crop production, or contribute to the precious wood industry, and the economic importance of the genus as a whole has probably been grossly underestimated. Some species, such as velvet apple (D. blancoi A. DC) in India and SE Asia, and jackalberry (D. mespiliformis Hochst. ex A. DC.), found throughout Africa, produce minor fruit crops, medicinal compounds, and precious ebony. Experimental research on Japanese persimmon has contributed to our knowledge of bioflavonoid (pigment) and polyphenolic compound (tannin) biosynthesis, and to our understanding of the relationship between genetics and the expression of astringency. It has been suggested that Japanese persimmon will be an important model organism for understanding traits related to tannin and bioflavonoid biosynthesis (Sablok et al. 2011).
 
World-scale molecular phylogenetic analyses of 119 species of Diospyros (Duangjai et al. 2009) suggest that the Mexican taxa (as well as Caribbean, Central American, and many of South American species) form a crown group with a stem that includes five Asian taxa. Members of the crown group are found in Indonesia, Sri Lanka, India, and China. Although Duangjai et al. (2009) sampled a large number of taxa, overall, there sample was a small proportion of the genus. None of the taxa we have sampled were included in their study, though the close relationship of the Mexican taxa we studied to their aforementioned crown group can be inferred, both morphologically and genetically (XXXX, XXXX, XXXX, XXXX). We performed high-throughput Illumina sequencing on four species of Diospyros that are endemic to Mexico, which belong to two of several species complexes proposed for the Central and Mesoamerica taxa (Provance and Sanders 200X, Provance et al 200X), where an estimated 30–35 species are present. The species we sampled include three members of the D. rosei complex, and one member of the D. salicifolia complex.
 
\subsection*{The Diospyros rosei complex}
Fruits of members of the D. rosei complex have 10–16 ovules, green when mature, or rarely red, and taste astringent taste when unripe. As the fruits ripen they turn brown to nearly black, astringency lessens or vanishes entirely, and the flesh softens and becomes sweet. The ripe fruit of some species can become pleasantly sweet and may develop exquisite subtleties in flavor. However, fruit taste varies among trees, and is influenced by post-harvest treatment. Furthermore, notions of palatability are subjective. The indehiscent berries are eaten and the seeds subsequently dispersed by various mammals (XXXX), and in at least one species (D. nigra), by freshwater turtles (XXXX). Floral features that characterize the D. rosei complex are white to cream-colored pendulous corollas with 4–8 obtuse to emarginate lobes, and a powerful jasmine-like scent. Female flowers have a 4–8 accrescent (mainly lengthening) sepals. The female corolla is usually shorter and broader relative to male flowers, and the pistil includes a 5–7-branched style. Male flowers are more elongated, with 16–32 stamens and a conspicuous pistillode.

The species most frequently cultivated in Central and Mesoamerican is D. nigra (often called zapote negro or zapote prieto), for which production in Mexico was estimated to be over 700 tons in 2001 by SAGARPA (Arellano-Gómez et al. 2005). Many other species are important minor fruit crops Mexico and Central America, though some that produce highly esteemed fruit are only consumed locally. The most important of these other species include D. riojae, D. conzattii, D. californica, D. sonorae, D. rosei, D. rekoi, D. morenoi, D. xolocotzii, D. texana, and D. oaxacana. The fruits of D. palmeri are used mainly by shepherd’s for goat forage, though human consumption of the species is known from an anecdote on one specimen label (per Wallnöfer). One of the authors (Provance) consumed fresh fruit of this species in near Arroyo Seco, San Luis Potosi, but found fresh ripe fruit from that population to be nearly impossible to ingest due to its high astringency and wretched flavor.
 
\subsection*{The Diospyros salicifolia complex}
Fruits of members of the \emph{D. salicifolia} complex have six ovules, and are when mature are green and taste astringent. The fruits of most species turn yellowish or golden brown (rarely red or brown in a couple Central American taxa). Their flesh is clear and gelatinous when ripe, and becomes reddish, translucent, and vitreous upon drying. The berries are indehiscent to vaguely dehiscent, or rarely conspicuously dehiscent. The seeds of many taxa are reddish, especially in life, and are dispersed by birds, which tear the fruits open and extract the seeds along with any adherent flesh. Evidently, the fruits are also eaten by various mammals, including carnivores (White 1978, pers. obs.). We have occasionally tasted the ripe fruits of several taxa, and have found that typically sweeten a small amount, but retain a great deal of astringency. Comments on some herbarium labels that suggest fruits are infrequently eaten by humans suggest that astringency may decrease to a greater extent in some individuals, or perhaps in some taxa that we have not sampled. The female flowering calyx is uniquely mitriform in the D. salicifolia complex, with 3-4 accrescent sepals (mostly widening, but also lengthening markedly in D. yucatanensis ssp. spectabilis). The corolla has 3–4 lanceolate to ovate lobes, and the tube is conspicuously sericeous, though with 3-4 glabrous zones around the base. The pistil includes 3 styles with bifid stigmas, and varying from being fully fused to free from the base. Male flowers have 9–14 stamens. Throughout the plant there are non-capitate hairs with fluid-filled lumens. This fluid is clear in life, but usually turns reddish upon drying. These hairs have rarely been reported in members of the D. rosei complex
 
Species sampled
Diospyros salicifolia complex:
Diospyros aequoris ssp. balsensis; one of three individuals germinated at UCRBG [UCRBG 09-084] from seeds collected near Arteaga (± 18.419, -102.1235, 630 m elevation), Michoacan, Mexico, X month year (Provance 9919, UCR). Seedlings were essentially grown to maturity in a greenhouse at UC Davis in preparation for experiments, then moved back to UCRBG. Voucher of progeny collected at UCRBG, X Nov 2013 (Provance XXXXXX, UCR, DAV).
 
Diospyros rosei complex:
Diospyros californica from near Todos Santos, Baja California Sur, Mexico (XXXXXXXX).
Diospyros rosei from Acahuato, Michoacan (XXXXXXXXX).
Diospyros palmeri fro Arroyo Seco, San Luis Potosi (XXXXX).
 
References:
Singh and Paul (2012) Drying of leaves of tendu (Diospyros melanoxylon) plants using a solar dryer with mirror booster. International Journal of Energy and Environment. 3(5): 799-808 [www.IJEE.IEEFoundation.org].

\end{document}